\section{Related Work}
\label{sec:related-work}


%##########################################################
%                     SEC RELATED WORK                    %
%##########################################################




\begin{table}[ht]
\centering
\normalsize
\caption{Comparison of communication paradigms.}
\label{tab:comp-paradigmas}


\resizebox{\linewidth}{!}{%
%\tiny
\begin{tabular}{|c|c|c|c|}
\hline \rowcolor[HTML]{cccccc}
\textbf{One-to-Many}       & \multicolumn{2}{c|}{\cellcolor[HTML]{cccccc}\textbf{Many-to-One}} & \textbf{Any-to-Any}              \\  \rowcolor[HTML]{cccccc}
\textbf{(Dissemination)}       & \multicolumn{2}{c|}{\cellcolor[HTML]{cccccc}\textbf{(Collection)}} & \textbf{(P2P)}      \\ \hline
\rowcolor[HTML]{cccccc}
\textit{\textbf{Unreliable}} & \textit{\textbf{Unreliable}}  & \textit{\textbf{Reliable}}       & \textit{\textbf{Reliable}} \\ \hline
Directed Diffusion~\cite{directedDiffusion}           & CTP~\cite{ctp}                          &                                  & AODV~\cite{AODV}, DYMO~\cite{dymo}                  \\ \cline{1-2} \cline{4-4}
(DIP, DRIP, DHV)~\cite{tinyos}               & MultiHopLQI~\cite{MultiHopLQI}                    &                                  & DSR~\cite{DSR}, Hydro~\cite{hydro}                      \\ \cline{1-2} \cline{4-4}
Deluge~\cite{deluge}                      & MintRoute~\cite{mintroute}                     & \multirow{-3}{*}{\textbf{XCTP}}  & RPL~\cite{RPL}                       \\ \hline
\end{tabular}
}


\end{table}


%\begin{table}[h]
%\centering
%\caption{Comparison of communication paradigms.}
%\label{tab:comp-paradigmas}
%\begin{tabular}{|c|c|c|c|}
%\hline
%\rowcolor[HTML]{C0C0C0} 
%\cellcolor[HTML]{C0C0C0}{\color[HTML]{000000} }                                                                                                 & \multicolumn{2}{c|}{\cellcolor[HTML]{C0C0C0}{\color[HTML]{000000} }}                                                                                              & \cellcolor[HTML]{C0C0C0}{\color[HTML]{000000} }                                                                                      \\
%\rowcolor[HTML]{C0C0C0} 
%\multirow{-2}{*}{\cellcolor[HTML]{C0C0C0}{\color[HTML]{000000} \textbf{\begin{tabular}[c]{@{}c@{}}One-to-Many\\ (Dissemination)\end{tabular}}}} & \multicolumn{2}{c|}{\multirow{-2}{*}{\cellcolor[HTML]{C0C0C0}{\color[HTML]{000000} \textbf{\begin{tabular}[c]{@{}c@{}}Many-to-One\\ (Collection)\end{tabular}}}}} & \multirow{-2}{*}{\cellcolor[HTML]{C0C0C0}{\color[HTML]{000000} \textbf{\begin{tabular}[c]{@{}c@{}}Any-to-Any\\ (P2P)\end{tabular}}}} \\ \hline
%\rowcolor[HTML]{C0C0C0} 
%{\color[HTML]{000000} \textit{\textbf{Unreliable}}}                                                                                             & {\color[HTML]{000000} \textit{\textbf{Unreliable}}}        & \multicolumn{1}{l|}{\cellcolor[HTML]{C0C0C0}{\color[HTML]{000000} \textit{\textbf{Reliable}}}}       & {\color[HTML]{000000} \textit{\textbf{Reliable}}}                                                                                    \\ \hline
%Directed Diffusion~\cite{directedDiffusion}                                                                                                                              & CTP~\cite{ctp}                                                        &                                                                                                      & AODV~\cite{AODV}, DYMO~\cite{dymo}, DSR~\cite{DSR} \\ \cline{1-2} \cline{4-4} 
%(DIP, DRIP, DHV)~\cite{tinyos}                                                                                                                                & MultiHopLQI~\cite{MultiHopLQI}                                        &                                                                                                      & Hydro~\cite{hydro}                                                                                                                                \\ \cline{1-2} \cline{4-4} 
%Deluge~\cite{deluge}                                                                                                                                          & MintRoute~\cite{mintroute}                                                  & \multirow{-3}{*}{XCTP}                                                                               & RPL~\cite{RPL}                                                                                                                                  \\ \hline
%\end{tabular}
%\end{table}







%\begin{table}[ht]
%\centering
%
%\caption{Comparison of communication paradigms.}
%\label{tab:comp-paradigmas}
%
%
%\resizebox{\linewidth}{!}{%
%\tiny
%\begin{tabular}{|c|c|c|c|}
%\hline \rowcolor[HTML]{cccccc} 
%\textbf{One-to-Many (Dissemination)}       & \multicolumn{2}{c|}{\cellcolor[HTML]{cccccc}\textbf{Many-to-One (Collection)}} & \textbf{Any-to-Any}              \\ \hline
%\rowcolor[HTML]{cccccc}
%\textit{\textbf{Unreliable}} & \textit{\textbf{Unreliable}}  & \textit{\textbf{Reliable}}       & \textit{\textbf{Reliable}} \\ \hline
%Directed Diffusion~\cite{directedDiffusion}           & CTP~\cite{ctp}                          &                                  & AODV~\cite{AODV}, DYMO~\cite{dymo}, DSR~\cite{DSR}                 \\ \cline{1-2} \cline{4-4}
%(DIP, DRIP, DHV)~\cite{tinyos}               & MultiHopLQI~\cite{MultiHopLQI}                    &                                  & Hydro~\cite{hydro}                      \\ \cline{1-2} \cline{4-4}
%Deluge~\cite{deluge}                      & MintRoute~\cite{mintroute}                     & \multirow{-3}{*}{\textbf{XCTP}}  & RPL~\cite{RPL}                       \\ \hline
%\end{tabular}
%}
%
%
%\end{table}




%\begin{table}[!t]
%\centering
%    \begin{tabular}{|c|c|c|}
%    \hline
%    Unreliable Collection & Reliable Collection & P2P       \\
%    \hline
%    CTP           & \multirow{3}{*}{XCTP}   &   AODV, Tymo  \\
%    %\cline{1-1} \cline{3-3}
%
%    MultiHopLQI   &                         &   DSR         \\
%    %\cline{1-1} \cline{3-3}
%
%    MintRoute     &                         &   Hydro       \\
%    \hline
%    \end{tabular}
%\caption{Tabela comparativa.}
%\end{table}


We present in Table~\ref{tab:comp-paradigmas} the main related
protocols. We classified them according to the communication
paradigm (\textit{any-to-any}, \textit{many-to-one},
\textit{one-to-many}). Table~\ref{tab:comp-paradigmas} shows that
\ac{xctp} is, to the best of our knowledge, the only Reliable
Collection protocol. In other words, it is a data collection
protocol that also allows unicast routes root-to-node. Besides that,
it offers an interface that facilitates the development of reliable
end-to-end transport protocols.

From the protocols presented in Table~\ref{tab:comp-paradigmas},
Directed Diffusion\cite{directedDiffusion}, (DIP, DRIP,
DHV)\cite{tinyos} are used for dissemination of small data packets
in the network. DIP, DRIP, DHV offers eventual consistency models
and use timers based on Trickle\cite{trickle}. DRIP treats each
information as a separated entity, which allows more control of when
and how fast the data will be disseminated. DIP and DHV treat data
as a group, meaning that control and dissemination parameters are
applied equally for all data.


\ac{ctp} and Deluge are related protocols. \ac{ctp} is a data
collection protocol that uses \ac{etx} metric to estimate the link
quality and route cost. Data and control packets are used to obtain
the link quality. \ac{xctp} is an extension of the \ac{ctp}
protocol. Besides creating unicast routes to a data collection
point, \ac{xctp} also creates unicast routes from the root to the
sensors. Deluge is a protocol that operates under the
\textit{one-to-many} paradigm, which has the objective to propagate
large amount of data, as is the case, when reprogramming the network
nodes.


%Hydro~\cite{hydro} and RPL~\cite{RPL} are protocols that aim at maintaining
%\textit{any-to-any} communication in \ac{wsn}. Hydro differs from our approach,
%which focus in creating unicast routes to exchange messages in both directions
%root-to-node and vice-versa. \ac{rpl} disseminates \ac{dao} messages through
%the network to announce routes for various destines inside a \ac{rpl} network.
%\ac{rpl} has two ways to send data, either up the tree or down the tree. 
%Upward routing has great scaling properties. But, downward routing does not scale 
%as well because the number of routes each node needs to have room for increases 
%linearly with network size. \ac{xctp}, different from \ac{rpl}, does not need 
%extra control messages to create downward routes, XCTP allows that non utilized downward routes be removed through of \acs{ttl}-based policies, avoids extra costs to keep state with \textit{peer-to-peer} routes that are under-utilized in WSN.
Hydro~\cite{hydro} and \ac{rpl}~\cite{RPL} are protocols that aim at
maintaining any-to-any communication in \ac{wsn}. Hydro differs from
our approach, which focus in creating unicast routes to exchange
messages in both directions root-to-node and vice-versa. Routing
Protocol for low-power and lossy networks (RPL) disseminates
Destination Advertisement Object (DAO) messages to announce routes
for each destination. While \ac{rpl} requires control packets to
create downward routes, XCTP does not have this overhead since XCTP
takes advantage of the data packets. XCTP also allows non utilized
downward routes to be removed through TTL-based policy, avoiding
memory overhead to store states with peer-to-peer routes that are
under-utilized in \ac{wsn}.


\ac{aodv}~\cite{AODV} and \ac{dsr}~\cite{DSR} are on-demand routing
protocols for \textit{any-to-any} communication. \ac{aodv} floods
the network with messages RREQ to build a path till the destination.
On the other hand, \ac{dsr} protocol uses the packet header to store
the route path. Unlike \ac{dsr}, our protocol does not store any
routing information in the packet header. \ac{aodv} protocol has
some similarity with \ac{xctp} in the strategy of storing the
reverse path. However \ac{xctp}, unlike \ac{aodv}, does not save
routes that are not reverse among the sensor nodes and the base
station. Dymo~\cite{dymo} is the \ac{aodv} successor, however it is
optimized for MANETs.


None of the protocols here related allow sending unicast messages in
 \textit{root-to-node} direction and vice versa, except those
 \textit{any-to-any} protocols that require large amount of information to be stored or control messages.
