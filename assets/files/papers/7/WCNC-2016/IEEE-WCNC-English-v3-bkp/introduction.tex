\section{Introduction}
\label{sec:introduction}

%##########################################################
%                     SEC INTRODUCTION                    %
%##########################################################

\acp{wsn} are composed of a large number of nodes with sensing,
computation, and wireless communication capability. These networks
have computing and communication energy constraints. Many
applications in \ac{wsn} need to transport large amount of data
(image, audio, video monitoring). These applications are not
tolerant to data loss, thus it is important to provide mechanisms to
reliable collect data.

The \acp{wsn} have the following communication paradigms:
\textit{many-to-one} (data collection), \textit{one-to-many} (data
dissemination), and a more complex way that enables communication
\textit{any-to-any}. First two paradigms allow the collection and
dissemination of data respectively. However, with routing on only
one direction, it is infeasible to build reliable mechanisms to
ensure the delivery of data end-to-end. \textit{Any-to-any}
communication paradigm allows communication between any pair of
nodes in the network, but adds more complexity and also requires
large amounts of memory to store all possible routes

In this work, we present \acf{xctp}, a routing protocol that is an
extension of the \ac{ctp}. \ac{ctp} creates a routing tree to
transfer data from one or more sensors to a root (sink) node. But,
\ac{ctp} does not create the reverse path between the root node and
sensors. This reverse path is important, for example, for feedback
commands or acknowledgment packets. \ac{xctp} enables communication
in both ways: sink to node and node to sink. \ac{xctp} requires low
storage of states and very low additional overhead in packets.

Our main contribution are as follows:
\begin{itemize}
     \item We propose \acl{xctp} (\ac{xctp}), which allows routing of messages in the reverse direction of \ac{ctp}, using a few extra memory to store reverse routes.
     \item We compare the performance of \ac{xctp}, \ac{aodv}, \ac{rpl}, and \ac{ctp}. In the experiments, \ac{xctp} proved to be more reliable, efficient, agile, and robust.
     \item We show that it is possible to implement reliable data transport protocol over \ac{xctp}.
\end{itemize}

\ac{ctp} optimizes data traffic towards the root thus achieves high
packet delivery rate. However, our \ac{xctp} approach goes beyond,
allowing bi-directional communication between sensor nodes and the
root. \ac{xctp} and \textit{any-to-any} routing protocols enable
reliable communication. However, \ac{xctp} reduces the cost to store
routes, since \ac{xctp} does not need to maintain routes to every
peer.

Our work is organized as follows. In the next section, we present
work related to \ac{xctp}. In Section~\ref{sec:problem}, we formally
define the problem being solved in this work. We describe \ac{xctp}
architecture in Section~\ref{sec:solution}. We compare \ac{xctp}
with \ac{aodv}, \ac{ctp} and present the simulation results in
Section{sec:evaluation}. Finally, we conclude in
Section~\ref{sec:conclusion}.
