In this work, we propose \ac{xctp}, a routing protocol that is an
extension of the \ac{ctp}. \ac{ctp} is the de-facto standard
collection routing protocol for Wireless Sensor Network (WSN).
\ac{ctp} creates a routing tree to transfer data from one or more
sensors to a root (sink) node. But, \ac{ctp} does not create the
reverse path between the root node and sensors. This reverse path is
important, for example, for feedback commands or acknowledgment
packets. \ac{xctp} enables communication in both ways: sink to node
and node to sink. \ac{xctp} accomplishes this task by exploring the
\ac{ctp} control plane packets. \ac{xctp} requires low storage
states and very low additional overhead in packets.  With the
reverse path, it is possible to implement reliable transport layer
protocols for \ac{wsn}. Thus, we designed \ac{tp}, a transport
protocol with \ac{arq} error-control on top of \ac{xctp}. We
implemented these protocols on TinyOS and evaluated on TOSSIM. We
compared \ac{xctp} with \ac{ctp}, \ac{rpl}, and \ac{aodv} protocols. We
conducted scalability and stress tests, evaluating them with
different loads and number of nodes. Our results shows that
\ac{xctp} is more reliable then \ac{ctp}, delivering $100\%$ of the
packets. \ac{xctp} sends fewer control packets than \ac{rpl}. \ac{xctp} is faster to recovery from network failures and also stores fewer states than \ac{aodv}, thus being efficient and agile.