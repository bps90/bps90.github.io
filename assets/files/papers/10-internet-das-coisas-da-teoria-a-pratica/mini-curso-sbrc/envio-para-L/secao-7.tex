% Crie um novo arquivo para cada parte do texto que for escrever.
% insira esse aquivo na posicao correta da secao. 

\section{Considerações Finais}  
\label{sec:CF}

No decorrer do capítulo, o leitor foi apresentado a diversos pontos da Internet das Coisas, passando por questões tanto teóricas quanto práticas da IoT. Por ser uma área extensa, ainda existem conteúdos que não foram discutidos ao longo do texto e, por esse motivo, alguns desses assuntos serão brevemente referenciados aqui e no conteúdo online\footnote{\scriptsize 
\textbf{\url{ 
http://homepages.dcc.ufmg.br/~bruno.ps/iot-tp-sbrc-2016/}}} como leitura complementar.

A IoT, neste momento, passa por questões que dizem respeito a sua forma, proprietários e regulamentações. Neste sentido, os próximos anos serão fundamentais para as definições e padronizações das tecnológicas para IoT. Neste sentido, empresas como Google, Apple e outras estão lançando seus próprios ecossistemas IoT, respectivamente \textit{Google Weave }e \textit{Apple HomeKit}. Entretanto, cada um possui protocolos, tecnologias de comunicação e padrões particulares. Isto torna a IoT não padronizada e consequentemente um ecossistema onde a IoT pode não prosperar. Assim, é preciso que a comunidade acadêmica e empresarial atentem-se às padronizações e na construção de um ecossistema favorável à IoT. Com isto será possível tornar os dispositivos \textit{Plug \& Play} e tornar a IoT livre de padrões proprietários. Em \cite{ishaq2013ietf}, os autores realizam uma revisão geral das padronizações em IoT.

Outra questão altamente relevante é a segurança para IoT. Este ponto, é uma das principais barreiras que impedem a efetiva adoção da IoT, pois os usuários estão preocupados com a violação dos seus dados. Deste modo, a segurança desempenha papel chave para a adoção da IoT. O escritor  Dominique Guinard resume, em seu artigo sobre políticas para IoT\footnote{\url{http://techcrunch.com/2016/02/25/the-politics-of-the-internet-of-things/}}, que ``\textit{A segurança dos objetos inteligentes é tão forte quanto seu enlace mais fraco}'', isto é, as soluções de segurança ainda não estão consolidadas, portanto soluções devem ser propostas e discutidas detalhadamente.

Neste trabalho foram descritos alguns dos princípios básicos da IoT através de uma abordagem teórico e prática. O aspecto dos objetos inteligentes e as tecnologias de comunicação foram abordados primeiramente. De forma complementar, discutiu-se sobre os \textit{software} que orquestram o funcionamento da IoT.  Foram realizadas duas atividades práticas para consolidar o conteúdo. Também apresentou-se o modo como gerenciar e analisar dados oriundos de dispositivos inteligentes, destacando as principais técnicas utilizadas. Além disso, apresentou-se uma perspectiva futura da IoT, como um meio para alcançar a computação ubíqua.

% FIM - Considerações finais