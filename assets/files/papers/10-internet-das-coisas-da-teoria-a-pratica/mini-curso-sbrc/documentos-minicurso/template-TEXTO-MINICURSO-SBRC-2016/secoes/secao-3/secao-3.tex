% Crie um novo arquivo para cada parte do texto que for escrever.
% insira esse aquivo na posicao correta da secao. 

\section{Da Teoria aos Softwares e Ambientes de Desenvolvimentos 
\textcolor{red}{BRUNO / LUCAS / BRUNA (MAX 15 pg)}}
\label{sec:dtsad}

  \subsection{O SW das Redes de Computadores convencionais X SW para 
dispositivos inteligentes. \textcolor{red}{BRUNO}}
    \begin{itemize}
      \item O SW deve ser pensado levando em considerações as limitações 
que em geral os dispositivos apresentam
    \end{itemize}

  \subsection{Paradigmas de comunicação dos dispositivos inteligentes 
\textcolor{red}{BRUNO/BRUNA}}
    \begin{itemize}
      \item Disseminação X Coleta X Par-a-Par
	\begin{itemize}
	  \item Como explorar os paradigmas para melhorar o desempenho dos 
dispositivos?
	  \item Muitos-para-um Ex: CTP, MultHopLQI...
	  \item Um-para-muitos: Direct Difusion, Deluxe, DIP/DRIP, 
CodeDRIP....
	  \item Qualquer-para-Qualquer: RPL, XCTP, Matrix
	  \item Localizá-los baseado em seu paradigma através de 
infográfico
	  \item Apresentar um comparativo que os diferencia
	\end{itemize}
    \end{itemize}

  \subsection{Modelos de conectividade Redes de Objetos inteligentes X IoT  
\textcolor{red}{BRUNO}}
    \begin{itemize}
      \item Autonomous Smart Objects networks - objetos que não requerem 
nenhuma conexão com a Intetnet (Ex: smart grids)
      \item Internet of Things - onde objetos inteligentes realmente estão 
conectados à Internet publica e podem ser acessados diretamente ou através de 
middlewares.
    \end{itemize}

  \subsection{Arquitetura TCP-UDP/IP para IoT. Ou o que não pôde ser 
reutilizado talvez possa ser adaptado \textcolor{red}{LUCAS}}
    \begin{enumerate}
      \item IP para Objetos Inteligentes? (Arquitetura TCP-UDP/6LoWPAN)
      \item Adaptações do IPv6 para chegar ao 6LoWPAN
      \item Pilhas TCP-UDP/6LowPAN reduzidas
      \begin{itemize}
	\item $\mu$IP e lwIP
      \end{itemize}
    \end{enumerate}
  
  \subsection{Ambientes de desenvolvimento \textcolor{red}{BRUNA}}
    \begin{itemize}
      \item Software geralmente deve mais enxuto
      \item Novas linguagens de programação
      \item Sistemas Operacionais
	\begin{enumerate}
	  \item Contiki
	  \item TinyOS
	\end{enumerate}
      \item Emuladores e Simuladores
	\begin{enumerate}
	  \item indicar qual é a diferença entre simulador X emulador
	  \item NS2/NS3
	  \item Cooja 
	  \item Tossim
	  \item OMNet++/Castalia
	  \item Sinalgo
	\end{enumerate}
    \end{itemize}
    
    \subsection{SW suas limitações e desafios que geram 
\textcolor{red}{LUCAS/BRUNA}}
      \begin{itemize}
	\item Problema do Gateway
	  \begin{itemize}
	    
	    \item Onde a ``inteligência'' deve ficar?
	      \begin{itemize}
		\item Se no Gateway outras questões surgem: se a conexão 
for perdida? e se for uma queda temporária? como implementar confirmações entre 
os dispositivos?)
		\item Se nos dispositivos: como enfrentar o trade-off com 
as limitações?
	      \end{itemize}
	    
	    \item Gateway fixo ou diferentes gateways? 
	    
	    \item Privacidade e Segurança
	      \begin{itemize}
		\item Ex: S e o gateway é um dispositivo de terceiros como
como manter a troca de informações de modo seguro? Se for um dispositivo de 
terceiros quais seriam os incentivos alguém transmita seus dados?
	      \end{itemize}
	      
	    \item IP móvel
	      \begin{itemize}
		\item Mobility Support in IPv6 RFC 6275
	      \end{itemize}
	  \end{itemize}
      \end{itemize}

% FIM - O Software e ambientes de desenvolvimento