% Crie um novo arquivo para cada parte do texto que for escrever.
% insira esse aquivo na posicao correta da secao. 

\section{Gerenciamento e Análise de Dados  \textcolor{red}{LUCAS/JOÃO/CLAYSON 
(MAX 15 pg)}}
\label{sec:GAD}

  \subsection{Técnicas para abstrair a heterogeneidade dos dispositivos 
\textcolor{red}{LUCAS}}
    \begin{itemize}
      \item CoAP, MQTT...
	\begin{itemize}
	  \item Um exemplo de abstração (RESTFul)
	\end{itemize}
      \item Ferramentas existentes (Plataformas de middleware)* 
\textcolor{red}{JOÃO}
	\begin{itemize}
	  \item Vai ocorrer alguma sobreposição com o minicurso passado 
(Plataformas para Internet das Coisas)
	\end{itemize}
    \end{itemize}

  \subsection{O manejo com dados oriundos dos dispositivos inteligentes 
\textcolor{red}{CLAYSON}}
    \begin{itemize}
      \item Formato dos dados (JSON, XML ...)
      \item Aspectos dos dados
      \begin{itemize}
	\item Espaços, Correlatos, Diferentes fontes, Imprecisos...
      \end{itemize}
    \end{itemize} 
      
  \subsection{Questões de pesquisa \textcolor{red}{JOÃO}}
    \begin{itemize}
      \item Qualidade dos dados (Estudo de caso)
      \item Fusão de dados (uma questão e 2 níveis de problemas)
      \begin{itemize}
	\item com o artigo que o professor passou para Bruno e 
João, ou seja fusão (Estudo de caso).
	\item in-networks
	\item ITS
      \end{itemize}
    \end{itemize}
% FIM - Gerenciamento e análise de dados