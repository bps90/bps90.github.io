% Crie um novo arquivo para cada parte do texto que for escrever.
% insira esse aquivo na posicao correta da secao. 

\section{Dos Dispositivos e Tecnologias de Comunicação 
\textcolor{red}{LUCAS / LOUREIRO (MAX 7 PG)}}
\label{sec:ddtc}

  \subsection{História dos objetos inteligentes \textcolor{red}{LUCAS}}
    \begin{itemize}
      \item História dos objetos inteligentes (um comparativo entre os 
elementos das redes de computadores convencionais e do que se chama hoje IoT) 

      \begin{itemize}
	\item Focar na capacidade de monitoramento (sensores), ou seja, a 
captação de dados é um grande diferencial destes novos elementos.
	\item Se possível construir uma ligação com os novos desafios que 
surgem 
  (seção Gerenciamento e Análise dos dados oriundos da IoT)
      \end{itemize}
    \end{itemize}
  
  \subsection{Arquiteturas básica dos dispositivos (processador, memória, 
bateria, sensores) \textcolor{red}{LUCAS}}
    
  \subsection{Tecnologias de comunicação \textcolor{red}{LUCAS}}
      \begin{itemize}
	\item Quais são as tecnologias de comunicação mais utilizadas nas 
redes   de dispositivos inteligentes? Cabeada x Sem fio?
	\item Infográfico diferenciando as tecnologias de comunicação
      \end{itemize}
  
  \subsection{Dispositivos suas limitações e desafios que geram 
\textcolor{red}{LOUREIRO}}
      \begin{itemize}
	\item Limitações (processamento, memória, energia)
	\item Custo, qualidade do HW, tamanho e outros...
	\item Levantar a discussão sobre conhecimento prévio das redes de 
computadores (tando de HW quanto de SW) e como devemos adaptá-los para esse 
novo mundo. O foco maior deve ser dado nas limitações do HW
	\item Energia como um grande desafio
	  \begin{itemize}
	    \item Energy Harvesting
	      \begin{itemize}
		\item O que é? Como fazer? Quais as direções?
		\item Exemplo da reunião: Dado que temos os dispositivos com 
capacidade de adquirir energia do ambiente e armazenar. Como programar as 
atividades que o dispositivo deve desempenhar dado o orçamento de energia 
(energy buget), isto é, como gastar a energia em função das atividades que se 
deve fazer?
	      \end{itemize}

	  \end{itemize}

      \end{itemize}
% FIM - Dos dispositivos e tecnologias de comunicação