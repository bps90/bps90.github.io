%
% Sample SBC book chapter
% This is a public-domain file.
% Charset: ISO8859-1 (latin-1) áéíóúç

% TEXTO de minicurso para o SBRC-2016

%    >> NOVAS DATAS IMPORTANTES <<
%
% Registro de propostas: 20/11/2015 (OK)
% Envio de Propostas: 06/12/2015 (OK)
% Comunicação dos resultados: 18/01/2016 (OK)
% Entrega dos capítulos: 21/03/2016



\documentclass{SBCbookchapter}
\usepackage[utf8]{inputenc}
\usepackage[T1]{fontenc}
\usepackage[brazilian]{babel}
\usepackage{graphicx}
\usepackage{color}
\usepackage{url}
%\usepackage[square, authoryear]{natbib}
\usepackage{enumitem}
%\setcounter{secnumdepth}{5}
%\setlength{\abovecaptionskip}{0cm}


\title{Internet das Coisas: da Teoria à Prática.}
%Connect and manage your things and data.
\author{Bruno P. Santos, Lucas A. M. Silva, Bruna S. Peres, Clayson S. F. S.  
Celes, João B. Borges Neto, Marcos Augusto M. Vieira, Luiz Filipe M. Vieira, 
Olga 
N. Goussevskaia e Antonio A. F. Loureiro}

\address{Departamento de Ciência da Computação -- Instituto de Ciências Exatas \linebreak
Universidade Federal de Minas Gerais (UFMG) -- Belo Horizonte, MG -- Brasil
\email{\{bruno.ps, lams, bperes, claysonceles, joaoborges, mmvieira, lfvieira, 
olga, loureiro\}@dcc.ufmg.br}
}

\hyphenation{de-mons-tra-ções}

\begin{document}

\maketitle

\begin{abstract}
% Lorem ipsum dolor sit amet, consectetur adipiscing elit. Proin ornare ex lectus. 
%  Proin ornare iaculis laoreet. Aenean auctor fringilla ornare. Aenean quis lorem 
% non purus feugiat malesuada. Phasellus euismod dignissim velit, sit amet euismod 
% nibh commodo eget. Curabitur leo ligula, egestas quis massa at, lacinia ornare 
% purus. Nunc placerat, magna eget scelerisque ultrices, leo arcu aliquam diam, ut 
% semper ex nulla nec nisl. Duis blandit, nibh non sollicitudin rutrum, turpis 
% magna venenatis felis, at tincidunt neque erat bibendum est. Morbi tempor ut 
% sapien eget faucibus. In venenatis, orci quis porta cursus, neque velit semper 
% mi, suscipit condimentum ipsum quam sed velit. Maecenas aliquet lorem justo, 
% dignissim dictum lorem pharetra ac. Fusce fringilla in augue id tincidunt. 
% Nullam nec sollicitudin ligula..
\end{abstract}

\begin{resumo}
\begin{otherlanguage}{brazilian}
A proliferação de objetos com capacidade de monitoramento, processamento e 
comunicação é crescente nos últimos anos. Diante disso, aparece o cenário de 
Internet das Coisas (Internet of Things - IoT) onde objetos podem se conectar à 
Internet e prover comunicação entre usuários, dispositivos (D2D), máquinas (M2M) 
e formarem novas aplicações. Várias questões teóricas e práticas surgem no 
desenvolvimento de IoT, por exemplo, como conectar esses objetos à Internet e 
como endereçar os objetos. Aliado a essas perguntas diversos desafios devem ser 
superados, por exemplo, como lidar com as restrições de processamento, largura 
de banda e energia dos dispositivos. Neste sentido, novos paradigmas de 
comunicação e roteamento podem ser explorados. Questões relacionadas ao 
endereçamento IP e como adaptá-lo precisam ser respondidas. Oportunidades de 
novas aplicações em uma rede de objetos inteligentes aparecem e, junto com essas 
aplicações, também surgem novos desafios. 

O objetivo deste minicurso é descrever o estado atual da Internet das Coisas da 
teoria à prática, e discutir desafios e questões de pesquisa. Através de uma 
abordagem crítica, é exposto uma visão geral da área, ilustrando soluções 
parciais e/ou totais propostas na literatura para as questões mencionadas, além 
de destacar os principais desafios e oportunidades que a área oferece. 

%%%%%%%%%%%%%% O texto abaixo foi comentado, pois o resumo ficou grande. %%%%%

% Iniciamos mostrando a evolução dos dispositivos embarcados e redes de sensores, 
% os quais são os blocos básicos de construção dos objetos inteligentes, sempre 
% pontuando os aspectos que evoluíram para formar o que é hoje a IoT. A seguir, 
% destacamos o potencial de aplicações possíveis sobre a rede de objetos 
% inteligentes e os desafios que devem ser enfrentados por essas aplicações, por 
% exemplo, dados imperfeitos, correlatos, inconsistentes e outros. Finalmente, 
% discutimos como trabalhar com a IoT, através de demonstração prática, empregando 
% os conceitos vistos em dispositivos reais e/ou exibindo as ferramentas como, por 
% exemplo, a instalação da pilha de protocolos de rede em dispositivos de baixa 
% potência (ex: 6LoWPAN/RPL e CoAP) e uso de ferramentas de gerenciamento 
% (Middlewares) e análise de dados. Além da descrição prática, discutimos os 
% desafios e as oportunidades de pesquisa na áreas de IoT.

\end{otherlanguage}
\end{resumo}
%os labels sao as iniciais de cada palavra do item

% Crie um novo arquivo para cada parte do texto que for escrever.
% insira esse aquivo na posicao correta da secao.

\section{Introdução \textcolor{red}{BRUNO (MAX 5 PG)}}
\label{sec:introducao}

  \begin{itemize}
    \item Breve introdução identificando os pontos que o minicurso aborda.
    \item O que são Objetos Inteligentes e qual sua posição no mundo das 
redes (Rede de Objetos Inteligentes  sem conexão com a rede IP X Objetos 
Inteligentes com IP)?
  \end{itemize}

  \subsection{Motivação}
    \begin{itemize}
      \item A história e a evolução dos dispositivos embarcados e RSSF
      \item A IoT não como um fim em si, é um meio para atingir a 
computação ubíqua e pervasiva
      \item A computação ubíqua e pervasiva
    \end{itemize}
  \subsection{Objetivos do minicurso}
  
  \subsection{Desafios para objetos inteligentes}
    \begin{itemize}
      \item Dispositivos, Tecnologias, Software e aplicações aqui colocados 
para criar um gancho para o restante do texto.
    \end{itemize}
    
  \subsection{Estrutura do minicurso.}
%FIM - introdução 

% Crie um novo arquivo para cada parte do texto que for escrever.
% insira esse aquivo na posicao correta da secao. 

\section{Dos Dispositivos e Tecnologias de Comunicação 
\textcolor{red}{LUCAS / LOUREIRO (MAX 7 PG)}}
\label{sec:ddtc}

  \subsection{História dos objetos inteligentes \textcolor{red}{LUCAS}}
    \begin{itemize}
      \item História dos objetos inteligentes (um comparativo entre os 
elementos das redes de computadores convencionais e do que se chama hoje IoT) 

      \begin{itemize}
	\item Focar na capacidade de monitoramento (sensores), ou seja, a 
captação de dados é um grande diferencial destes novos elementos.
	\item Se possível construir uma ligação com os novos desafios que 
surgem 
  (seção Gerenciamento e Análise dos dados oriundos da IoT)
      \end{itemize}
    \end{itemize}
  
  \subsection{Arquiteturas básica dos dispositivos (processador, memória, 
bateria, sensores) \textcolor{red}{LUCAS}}
    
  \subsection{Tecnologias de comunicação \textcolor{red}{LUCAS}}
      \begin{itemize}
	\item Quais são as tecnologias de comunicação mais utilizadas nas 
redes   de dispositivos inteligentes? Cabeada x Sem fio?
	\item Infográfico diferenciando as tecnologias de comunicação
      \end{itemize}
  
  \subsection{Dispositivos suas limitações e desafios que geram 
\textcolor{red}{LOUREIRO}}
      \begin{itemize}
	\item Limitações (processamento, memória, energia)
	\item Custo, qualidade do HW, tamanho e outros...
	\item Levantar a discussão sobre conhecimento prévio das redes de 
computadores (tando de HW quanto de SW) e como devemos adaptá-los para esse 
novo mundo. O foco maior deve ser dado nas limitações do HW
	\item Energia como um grande desafio
	  \begin{itemize}
	    \item Energy Harvesting
	      \begin{itemize}
		\item O que é? Como fazer? Quais as direções?
		\item Exemplo da reunião: Dado que temos os dispositivos com 
capacidade de adquirir energia do ambiente e armazenar. Como programar as 
atividades que o dispositivo deve desempenhar dado o orçamento de energia 
(energy buget), isto é, como gastar a energia em função das atividades que se 
deve fazer?
	      \end{itemize}

	  \end{itemize}

      \end{itemize}
% FIM - Dos dispositivos e tecnologias de comunicação
% Crie um novo arquivo para cada parte do texto que for escrever.
% insira esse aquivo na posicao correta da secao. 

\section{Da Teoria aos Softwares e Ambientes de Desenvolvimentos 
\textcolor{red}{BRUNO / LUCAS / BRUNA (MAX 15 pg)}}
\label{sec:dtsad}

  \subsection{O SW das Redes de Computadores convencionais X SW para 
dispositivos inteligentes. \textcolor{red}{BRUNO}}
    \begin{itemize}
      \item O SW deve ser pensado levando em considerações as limitações 
que em geral os dispositivos apresentam
    \end{itemize}

  \subsection{Paradigmas de comunicação dos dispositivos inteligentes 
\textcolor{red}{BRUNO/BRUNA}}
    \begin{itemize}
      \item Disseminação X Coleta X Par-a-Par
	\begin{itemize}
	  \item Como explorar os paradigmas para melhorar o desempenho dos 
dispositivos?
	  \item Muitos-para-um Ex: CTP, MultHopLQI...
	  \item Um-para-muitos: Direct Difusion, Deluxe, DIP/DRIP, 
CodeDRIP....
	  \item Qualquer-para-Qualquer: RPL, XCTP, Matrix
	  \item Localizá-los baseado em seu paradigma através de 
infográfico
	  \item Apresentar um comparativo que os diferencia
	\end{itemize}
    \end{itemize}

  \subsection{Modelos de conectividade Redes de Objetos inteligentes X IoT  
\textcolor{red}{BRUNO}}
    \begin{itemize}
      \item Autonomous Smart Objects networks - objetos que não requerem 
nenhuma conexão com a Intetnet (Ex: smart grids)
      \item Internet of Things - onde objetos inteligentes realmente estão 
conectados à Internet publica e podem ser acessados diretamente ou através de 
middlewares.
    \end{itemize}

  \subsection{Arquitetura TCP-UDP/IP para IoT. Ou o que não pôde ser 
reutilizado talvez possa ser adaptado \textcolor{red}{LUCAS}}
    \begin{enumerate}
      \item IP para Objetos Inteligentes? (Arquitetura TCP-UDP/6LoWPAN)
      \item Adaptações do IPv6 para chegar ao 6LoWPAN
      \item Pilhas TCP-UDP/6LowPAN reduzidas
      \begin{itemize}
	\item $\mu$IP e lwIP
      \end{itemize}
    \end{enumerate}
  
  \subsection{Ambientes de desenvolvimento \textcolor{red}{BRUNA}}
    \begin{itemize}
      \item Software geralmente deve mais enxuto
      \item Novas linguagens de programação
      \item Sistemas Operacionais
	\begin{enumerate}
	  \item Contiki
	  \item TinyOS
	\end{enumerate}
      \item Emuladores e Simuladores
	\begin{enumerate}
	  \item indicar qual é a diferença entre simulador X emulador
	  \item NS2/NS3
	  \item Cooja 
	  \item Tossim
	  \item OMNet++/Castalia
	  \item Sinalgo
	\end{enumerate}
    \end{itemize}
    
    \subsection{SW suas limitações e desafios que geram 
\textcolor{red}{LUCAS/BRUNA}}
      \begin{itemize}
	\item Problema do Gateway
	  \begin{itemize}
	    
	    \item Onde a ``inteligência'' deve ficar?
	      \begin{itemize}
		\item Se no Gateway outras questões surgem: se a conexão 
for perdida? e se for uma queda temporária? como implementar confirmações entre 
os dispositivos?)
		\item Se nos dispositivos: como enfrentar o trade-off com 
as limitações?
	      \end{itemize}
	    
	    \item Gateway fixo ou diferentes gateways? 
	    
	    \item Privacidade e Segurança
	      \begin{itemize}
		\item Ex: S e o gateway é um dispositivo de terceiros como
como manter a troca de informações de modo seguro? Se for um dispositivo de 
terceiros quais seriam os incentivos alguém transmita seus dados?
	      \end{itemize}
	      
	    \item IP móvel
	      \begin{itemize}
		\item Mobility Support in IPv6 RFC 6275
	      \end{itemize}
	  \end{itemize}
      \end{itemize}

% FIM - O Software e ambientes de desenvolvimento
% Crie um novo arquivo para cada parte do texto que for escrever.
% insira esse aquivo na posicao correta da secao. 

\section{IoT na Prática \textcolor{red}{BRUNO/BRUNA (MAX 5 pg)}} 
\label{sec:ITP}

  \begin{itemize}
    \item Definir quais serão os experimentos.
    \begin{itemize}
      \item A proposta inicial seria:
      \begin{itemize}
	\item Instalar os códigos RPL/6LowPAN de um ou dos SOs TinyOS e 
Contiki, porém  existe o problema de como explicar tal instalação no texto que 
pode ser extenso. Além da instalação, será apresentada uma demonstração 
consultas aos sensores dos nós através de requisições CoAP
	\item Realizar uma consulta em uma plataforma middleware tal como o 
João utiliza a Xyvely
	\item OBS: vale notar que as demonstrações serão 
realizadas ao longo da apresentação e não em um momento específico.
	\item OBS: podemos criar um vídeo para exemplificar de forma mais 
rápida e confiável, mas ainda assim seria interessante levar os motes reais 
para validar o conteúdo do vídeo.
      \end{itemize}
    \end{itemize}
  \end{itemize}
% FIM - IoT na prática
% Crie um novo arquivo para cada parte do texto que for escrever.
% insira esse aquivo na posicao correta da secao. 

\section{Gerenciamento e Análise de Dados  \textcolor{red}{LUCAS/JOÃO/CLAYSON 
(MAX 15 pg)}}
\label{sec:GAD}

  \subsection{Técnicas para abstrair a heterogeneidade dos dispositivos 
\textcolor{red}{LUCAS}}
    \begin{itemize}
      \item CoAP, MQTT...
	\begin{itemize}
	  \item Um exemplo de abstração (RESTFul)
	\end{itemize}
      \item Ferramentas existentes (Plataformas de middleware)* 
\textcolor{red}{JOÃO}
	\begin{itemize}
	  \item Vai ocorrer alguma sobreposição com o minicurso passado 
(Plataformas para Internet das Coisas)
	\end{itemize}
    \end{itemize}

  \subsection{O manejo com dados oriundos dos dispositivos inteligentes 
\textcolor{red}{CLAYSON}}
    \begin{itemize}
      \item Formato dos dados (JSON, XML ...)
      \item Aspectos dos dados
      \begin{itemize}
	\item Espaços, Correlatos, Diferentes fontes, Imprecisos...
      \end{itemize}
    \end{itemize} 
      
  \subsection{Questões de pesquisa \textcolor{red}{JOÃO}}
    \begin{itemize}
      \item Qualidade dos dados (Estudo de caso)
      \item Fusão de dados (uma questão e 2 níveis de problemas)
      \begin{itemize}
	\item com o artigo que o professor passou para Bruno e 
João, ou seja fusão (Estudo de caso).
	\item in-networks
	\item ITS
      \end{itemize}
    \end{itemize}
% FIM - Gerenciamento e análise de dados
% Crie um novo arquivo para cada parte do texto que for escrever.
% insira esse aquivo na posicao correta da secao. 

\section{IoT como o meio para a Computação Ubíqua e pervasiva 
\textcolor{red}{LOUREIRO (MAX 2 PG)} }
\label{sec:DITCUP}

  \begin{enumerate}
    \item Exemplos de Aplicações (Automação residencial, Smart Cities, 
Urban Networks, Monitoramento de Saúde)
    \item Definição de entidades
      \begin{itemize}
	\item Diferentes tipos
	\item Diferentes contextos (Físico e Lógicos)
      \end{itemize}
    \item Aquisição de contexto através dos dados dos dispositivos 
inteligentes
    \item Elementos para monitoramento (sensores)
    \item O papel da ``\textit{Cloud Computing}''
  \end{enumerate}
% FIM - IoT como o meio para a Computação Ubíqua e pervasiva
% Crie um novo arquivo para cada parte do texto que for escrever.
% insira esse aquivo na posicao correta da secao. 

\section{Considerações Finais}  
\label{sec:CF}

No decorrer do capítulo, o leitor foi apresentado a diversos pontos da Internet das Coisas, passando por questões tanto teóricas quanto práticas da IoT. Por ser uma área extensa, ainda existem conteúdos que não foram discutidos ao longo do texto e, por esse motivo, alguns desses assuntos serão brevemente referenciados aqui e no conteúdo online\footnote{\scriptsize 
\textbf{\url{ 
http://homepages.dcc.ufmg.br/~bruno.ps/iot-tp-sbrc-2016/}}} como leitura complementar.

A IoT, neste momento, passa por questões que dizem respeito a sua forma, proprietários e regulamentações. Neste sentido, os próximos anos serão fundamentais para as definições e padronizações das tecnológicas para IoT. Neste sentido, empresas como Google, Apple e outras estão lançando seus próprios ecossistemas IoT, respectivamente \textit{Google Weave }e \textit{Apple HomeKit}. Entretanto, cada um possui protocolos, tecnologias de comunicação e padrões particulares. Isto torna a IoT não padronizada e consequentemente um ecossistema onde a IoT pode não prosperar. Assim, é preciso que a comunidade acadêmica e empresarial atentem-se às padronizações e na construção de um ecossistema favorável à IoT. Com isto será possível tornar os dispositivos \textit{Plug \& Play} e tornar a IoT livre de padrões proprietários. Em \cite{ishaq2013ietf}, os autores realizam uma revisão geral das padronizações em IoT.

Outra questão altamente relevante é a segurança para IoT. Este ponto, é uma das principais barreiras que impedem a efetiva adoção da IoT, pois os usuários estão preocupados com a violação dos seus dados. Deste modo, a segurança desempenha papel chave para a adoção da IoT. O escritor  Dominique Guinard resume, em seu artigo sobre políticas para IoT\footnote{\url{http://techcrunch.com/2016/02/25/the-politics-of-the-internet-of-things/}}, que ``\textit{A segurança dos objetos inteligentes é tão forte quanto seu enlace mais fraco}'', isto é, as soluções de segurança ainda não estão consolidadas, portanto soluções devem ser propostas e discutidas detalhadamente.

Neste trabalho foram descritos alguns dos princípios básicos da IoT através de uma abordagem teórico e prática. O aspecto dos objetos inteligentes e as tecnologias de comunicação foram abordados primeiramente. De forma complementar, discutiu-se sobre os \textit{software} que orquestram o funcionamento da IoT.  Foram realizadas duas atividades práticas para consolidar o conteúdo. Também apresentou-se o modo como gerenciar e analisar dados oriundos de dispositivos inteligentes, destacando as principais técnicas utilizadas. Além disso, apresentou-se uma perspectiva futura da IoT, como um meio para alcançar a computação ubíqua.

% FIM - Considerações finais

\nocite{*}
\bibliographystyle{sbc}
\bibliography{refs}

\end{document}

