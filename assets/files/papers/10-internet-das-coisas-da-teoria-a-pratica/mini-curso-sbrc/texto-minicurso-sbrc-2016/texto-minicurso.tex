%
% Sample SBC book chapter
% This is a public-domain file.
% Charset: ISO8859-1 (latin-1) áéíóúç

% Proposta de minicurso para o SBRC-2016

%    >> DATAS IMPORTANTES <<
%
% Registro de propostas: 20/11/2015
% Envio de Propostas: 27/11/2015
% Comunicação dos resultados: 18/01/2016
% Entrega dos capítulos: 21/03/2016

%    >> NOVAS DATAS IMPORTANTES <<
%
% Registro de propostas: 20/11/2015
% Envio de Propostas: 06/12/2015
% Comunicação dos resultados: 18/01/2016
% Entrega dos capítulos: 21/03/2016



\documentclass{SBCbookchapter}
\usepackage[utf8]{inputenc}
\usepackage[T1]{fontenc}
\usepackage[brazilian]{babel}
\usepackage{graphicx}
\usepackage{color}
\usepackage{url}
%\usepackage[square, authoryear]{natbib}
\usepackage{enumitem}
%\setcounter{secnumdepth}{5}
%\setlength{\abovecaptionskip}{0cm}


\title{Internet das Coisas: da Teoria à Prática.}
%Connect and manage your things and data.
\author{Bruno P. Santos, Lucas A. M. Silva, Bruna S. Peres, Clayson S. F. S.  
Celes, João B. Borges Neto, Marcos Augusto M. Vieira, Luiz Filipe M. Vieira, 
Olga 
N. Goussevskaia e Antonio A. F. Loureiro}

\address{Departamento de Ciência da Computação -- Instituto de Ciências Exatas \linebreak
Universidade Federal de Minas Gerais (UFMG) -- Belo Horizonte, MG -- Brasil
\email{\{bruno.ps, lams, bperes, claysonceles, joaoborges, mmvieira, lfvieira, 
olga, loureiro\}@dcc.ufmg.br}
}

\hyphenation{de-mons-tra-ções}

\begin{document}

\maketitle

% \begin{abstract}
% \textcolor{red}{Máximo de 10~linhas cada (abstract e resumo)}.\\
% Lorem ipsum dolor sit amet, consectetur adipiscing elit. Proin ornare ex lectus. 
%  Proin ornare iaculis laoreet. Aenean auctor fringilla ornare. Aenean quis lorem 
% non purus feugiat malesuada. Phasellus euismod dignissim velit, sit amet euismod 
% nibh commodo eget. Curabitur leo ligula, egestas quis massa at, lacinia ornare 
% purus. Nunc placerat, magna eget scelerisque ultrices, leo arcu aliquam diam, ut 
% semper ex nulla nec nisl. Duis blandit, nibh non sollicitudin rutrum, turpis 
% magna venenatis felis, at tincidunt neque erat bibendum est. Morbi tempor ut 
% sapien eget faucibus. In venenatis, orci quis porta cursus, neque velit semper 
% mi, suscipit condimentum ipsum quam sed velit. Maecenas aliquet lorem justo, 
% dignissim dictum lorem pharetra ac. Fusce fringilla in augue id tincidunt. 
% Nullam nec sollicitudin ligula..
% \end{abstract}

\begin{resumo}
\begin{otherlanguage}{brazilian}
% Máximo de 10~linhas cada (abstract e resumo).
A proliferação de objetos com capacidade de monitoramento, processamento e 
comunicação é crescente nos últimos anos. Diante disso, aparece o cenário de 
Internet das Coisas (Internet of Things - IoT) onde objetos podem se conectar à 
Internet e prover comunicação entre usuários, dispositivos (D2D), máquinas (M2M) 
e formarem novas aplicações. Várias questões teóricas e práticas surgem no 
desenvolvimento de IoT, por exemplo, como conectar esses objetos à Internet e 
como endereçar os objetos. Aliado a essas perguntas diversos desafios devem ser 
superados, por exemplo, como lidar com as restrições de processamento, largura 
de banda e energia dos dispositivos. Neste sentido, novos paradigmas de 
comunicação e roteamento podem ser explorados. Questões relacionadas ao 
endereçamento IP e como adaptá-lo precisam ser respondidas. Oportunidades de 
novas aplicações em uma rede de objetos inteligentes aparecem e, junto com essas 
aplicações, também surgem novos desafios. 

O objetivo deste minicurso é descrever o estado atual da Internet das Coisas da 
teoria à prática, e discutir desafios e questões de pesquisa. Através de uma 
abordagem crítica, é exposto uma visão geral da área, ilustrando soluções 
parciais e/ou totais propostas na literatura para as questões mencionadas, além 
de destacar os principais desafios e oportunidades que a área oferece. 

%%%%%%%%%%%%%% O texto abaixo foi comentado, pois o resumo ficou grande. %%%%%

% Iniciamos mostrando a evolução dos dispositivos embarcados e redes de sensores, 
% os quais são os blocos básicos de construção dos objetos inteligentes, sempre 
% pontuando os aspectos que evoluíram para formar o que é hoje a IoT. A seguir, 
% destacamos o potencial de aplicações possíveis sobre a rede de objetos 
% inteligentes e os desafios que devem ser enfrentados por essas aplicações, por 
% exemplo, dados imperfeitos, correlatos, inconsistentes e outros. Finalmente, 
% discutimos como trabalhar com a IoT, através de demonstração prática, empregando 
% os conceitos vistos em dispositivos reais e/ou exibindo as ferramentas como, por 
% exemplo, a instalação da pilha de protocolos de rede em dispositivos de baixa 
% potência (ex: 6LoWPAN/RPL e CoAP) e uso de ferramentas de gerenciamento 
% (Middlewares) e análise de dados. Além da descrição prática, discutimos os 
% desafios e as oportunidades de pesquisa na áreas de IoT.

\end{otherlanguage}
\end{resumo}


% 1. Dados de Identificação
%   1.1 Título do minicurso
%   1.2 Autor(es): instituição(ções), endereço(s), telefone(s), e-mail(s)
%   1.3 Indicação do autor que apresentará o curso
% 2. Dados Gerais
%   2.1. Objetivos do curso e tratamento dado ao tema (ex: teórico ou prático,  
% apanhado geral de resultados ou aprofundamento de aspectos específicos, 
% apresentação ou comparação de tecnologias, formação de novas habilidades ou 
% informação, etc.)
%   2.2. Perfil do público alvo
% 3. Estrutura prevista do texto (esboço do curso, tópicos/capítulos cobertos)
% 4. Para cada tópico/capítulo previsto no item 3, informar um resumo do 
% conteúdo a ser abordado e previsão do número de páginas
% 5. Bibliografia principal utilizada na preparação do curso (máximo de 1 página)
% 6. Curriculum Vitae resumido dos autores (máximo de 1 página por autor)


\section{Indicação dos autores apresentadores}
Bruno P. Santos, Clayson S. F. S. Celes, Lucas A. M. Silva, Antonio A. F. 
Loureiro.

\section{Dados gerais}

\subsection{Motivação}
%OK
% Comentários - Clayson
% Quando se refere a uma ideia ou oração já mencionada usar *esse* e nao 
% *este*. Não é isso?

Ao conectar objetos com diferentes recursos a uma rede, potencializa-se o 
surgimento de novas aplicações. Neste sentido, conectar esses objetos à 
Internet significa criar a Internet das Coisas (Internet of Things - IoT). Na 
IoT, os objetos podem prover comunicação entre usuários, dispositivos (D2D) e 
máquinas (M2M). Com isto emerge uma nova gama de aplicações, tais como coleta de 
dados de pacientes e monitoramento de idosos (\textit{healthcare at home}), 
sensoriamento de ambientes de difícil acesso e inóspitos, entre outras.
% OK
% Comentários - Clayson
% 1. Ao conectar objetos com diferentes recursos computacionais a uma rede,  
% potencializa-se o surgimento de novas aplicações.
% 2. com isto emerge uma gama de aplicações, tais como coleta de dados de pacientes 
% e monitoramento de idosos (\textit{healthcare at home}), sensoriamento de 
% ambientes de difícil acesso e inóspitos, entre outras.
% 3. Eu não colocaria Internet (IP), apenas Internet.

Entretanto, ao passo que a possibilidade de novas aplicações é crescente, os 
desafios de conectar os objetos à Internet surgem. Naturalmente, os objetos são
heterogêneos, isto é, divergem em implementação, recursos e qualidade e, em 
geral, apresentam limitações tais como: energia, capacidade de processamento, 
memória e comunicação. Questões tanto teóricas quanto práticas surgem, por 
exemplo, como prover endereçamento aos dispositivos, como encontrar rotas de 
alta vazão e melhor taxa de entrega ao passo que o baixo consumo de energia seja 
mantido. Deste modo, fica evidente a necessidade da adaptação dos protocolos 
existentes (por exemplo o IP). Além disso, os paradigmas de comunicação e 
roteamento devem ser explorados para permitir prover os serviços fundamentais 
para formar novas aplicações. 

Novos desafios surgem ao passo que são criadas novas aplicações para IoT. Os 
dados providos pelos objetos agora podem apresentar imperfeições (calibragem do 
sensor), inconsistências (fora de ordem, \textit{outliers}), serem de diferentes 
tipos (gerados por pessoas, sensores físicos, fusão de dados). Assim, as 
aplicações e algoritmos devem ser capazes de lidar com esses desafios sobre os 
dados. Outro exemplo diz respeito ao nível de confiança sobre os dados obtidos 
dos dispositivos da IoT e como/onde  podemos empregar esses dados em 
determinados cenários. Deste modo, os desafios impostos por essas novas 
aplicações sobre a IoT devem ser explorados e soluções devem ser propostas.
%OK
% Comentários - Clayson
% como podemos empregar esses dados em determinados cenários.

%------------- Versão antiga ----------------------
% \textcolor[rgb]{.4,.4,.4}{
% Os autores do presente minicurso, notam a falta de documentação e/ou 
% trabalhos que mostram como implementar \textit{Internet of Things} (IoT) na 
% prática, em especial, no nosso idioma. Na literatura, é fácil encontrar 
% documentos que descrevem os dispositivos e softwares utilizados, entretanto não 
% é fácil encontrar os softwares e documentos que mostram como instalar e 
% desenvolver para IoT. Apesar de encontrarmos trabalhos que descrevem 
% plataformas de \textit{middleware} capazes de gerenciar as informações advindas 
% dos dispositivos IoT, essas plataformas abstraem as complexidades das bases da 
% IoT. Para aqueles que desejam entender as bases e implementar uma ``IoT 
% própria'' a curva de aprendizado é dispendiosa, visto que é necessária diversas 
% buscas para entender os conceitos, dispositivos e softwares a serem utilizados, 
% além de todo o trabalho para a instalação e escolha das ferramentas de 
% desenvolvimento. Diante deste cenário, nos vemos motivados a abrandar essa curva 
% aprendizado para entender e implementar uma IoT. }

% \textcolor[rgb]{.4,.4,.4}{
% Diferente dos minicursos anteriormente apresentados no SBRC, nossa proposta 
% visa mostrar como ligar a teoria, a prática, as aplicações e serviços existentes 
% na IoT. Para evidenciar as diferenças dessa proposta, destacamos que no 
% minicurso \textit{Plataformas para a Internet das Coisas} de 2015 e \textit{Web 
% das coisas: conectando dispositivos físicos ao mundo digital} de 2011, pouca 
% atenção foi dada a quais dispositivos são empregados na IoT, suas limitações e 
% desafios, %em nossa proposta esses quesitos serão abordados.
% questões que serão abordadas em nossa proposta.Em 2010, foi apresentado o 
% minicurso \textit{Redes de Sensores Aquáticas}, o qual teve enfoque em redes de 
% sensores e redes aquáticas destacando tanto hardware quanto protocolos da época. 
% Entretanto, não foi abordado os trabalhos que hoje compõem os principais 
% protocolos para IoT, por exemplo, \textit{IPv6 in Low-Power Wireless Personal 
% Area Networks} (6LoWPAN) e o protocolo roteamento \textit{IPV6 Routing Protocol 
% for Low-Power and Lossy Networks} (RPL). Hoje esses protocolos estão mais 
% consolidados e, por isso, temos a intenção de comentar e contextualizar estes e 
% outros protocolos ao longo do minicurso proposto. O minicurso 
% \textit{Arquiteturas para Redes de Sensores Sem Fio} de 2004, por sua vez, 
% possui uma perspectiva introdutória e de caracterização das redes de sensores 
% sem fio. Diferente deste minicurso, nós pretendemos contemplar desde uma 
% introdução a IoT, os dispositivos e \textit{softwares} de implementação e 
% desenvolvimento, realizar exercícios práticos, chegando até conceitos mais 
% avançados como as padronizações (MQTT e CoAP) que viabilizam a construção das 
% plataformas de \textit{middleware}, bem como questões de pesquisa na área.}

\subsection{Objetivos do curso}
%OK
% Substituir: dando enfoque aos seus blocos básicos de construção e especificidades.
%        por: com enfoque em seus blocos básicos de construção e especificidades.
O objetivo deste minicurso é descrever o estado atual da Internet das Coisas da 
teoria à prática. Para alcançar esse objetivo, a IoT será contextualizada na 
história das redes de computadores, com enfoque em seus blocos básicos de 
construção e especificidades. Com base nesta contextualização e por meio 
de uma abordagem crítica são expostos os desafios e questões de pesquisa que a 
IoT impõe.

%OK
%Substituir: "dentro de"
%       por: "em"
Os desafios e questões de pesquisa que o minicurso aborda podem ser divididos 
em quatro categorias:

\begin{enumerate}[label=\roman*)]
  \item \textbf{Objetos IoT}: são aqueles que envolvem blocos básicos de 
construção dos objetos inteligentes capacitando-os a operar na IoT. Em geral, as 
características desejáveis para os dispositivos são: apresentar baixo consumo de 
energia; possibilidade de comunicação e; tamanho reduzido;
  
  \item \textbf{Redes de comunicação}: dizem respeito a como conectar os 
dispositivos à infraestrutura existente (IP), como devem ser explorados os 
paradigmas de comunicação e roteamento para prover protocolos otimizados para 
IoT que levem em conta as limitações dos dispositivos, escalabilidade e outras;

%OK
%Substituir:
%Esses dispositivos por serem construídos por diferentes fabricantes 
% apresentam, como consequência, especificidades...
%Por:
% Por serem construídos por diferentes fabricantes, esses dispositivos 
%apresentam especificidades....
  \item \textbf{Padronizações}: destacam o impacto das padronizações sobre 
a intrínseca heterogeneidade dos objetos. Por serem construídos por diferentes 
fabricantes, esses dispositivos apresentam especificidades, por exemplo, alta 
precisão para alguns sensores e baixa para outros. Por isso, padronizar o 
acesso aos recursos e trabalhar com dados oriundos da IoT é tão desafiador. 

  \item \textbf{Novas aplicações em IoT}: neste quesito se enquadram as 
questões referentes às novas aplicações sobre IoT, dificuldades encontradas 
para tratar informações advindas dos objetos inteligentes. Assim, almejamos 
descrever quais são os principais problemas encontrados ao se implementar sobre 
IoT e quais os desafios que devem ser enfrentados. 
\end{enumerate}

\noindent Outro propósito deste curso é apresentar a visão geral da área 
através de soluções presentes na literatura para as questões acima citadas. As 
soluções discutidas ao longo do minicurso são demonstradas, pelos 
apresentadores, seguindo o estilo prático. O principal objetivo da exposição 
das soluções práticas é consolidar os conceitos vistos ao longo do curso. Para 
isso, exibiremos  como anexar a um dispositivo IoT de baixa potência a pilha 
básica de protocolos e uma aplicação (Ex: 6LOWPAN/RPL e CoAP). Também será 
exibida uma prática de como utilizar ferramentas de gerenciamento 
(\textit{middleware}) e análise de dados. Portanto, os leitores e ouvintes 
podem experimentar a implantação de uma rede de dispositivos inteligentes, além 
de conhecer o estado da arte em IoT apreciando as questões e desafios do 
hardware e software utilizados.

% \textcolor[rgb]{.4,.4,.4}{
% Este minicurso tem por objetivo apresentar a Internet das Coisas pontuando sua 
% história, blocos básicos de construção (tanto \textit{hardware} quanto 
% \textit{software}), implementação e validação prática dos conceitos discutidos. 
% Diante de um cenário IoT, visto na prática, também temos como meta mostrar como 
% gerenciar os dispositivos e dados, além de pontuar as principais oportunidades 
% de inovação e questões de pesquisa na área de IoT. Ao final do minicurso, o 
% participante terá compreendido o processo de evolução e conceitos que englobam a 
% implementação IoT, apreciando os desafios dessa rede implementada com 
% dispositivos heterogêneos, comunicação sem fio, diferentes sensores, variados 
% protocolos, plataformas de~\textit{middlewares} para gerenciamento de dados, bem 
% como discutir quais são as tendências para o futuro da IoT.
% }

\subsubsection{Abordagem do minicurso}

A abordagem empregada neste curso será no estilo~\textit{teórico-prática}. No 
primeiro momento, uma introdução e contextualização dos leitores e participantes 
será realizada, identificando os blocos iniciais para a construção e 
desenvolvimento da IoT. No segundo momento, mostraremos uma implementação 
prática\footnote{Vale ressaltar que a demonstração prática requer uso 
de~\textit{hardware}, portanto estamos avaliando a viabilidade de execução 
prática com uso de nós sensores TelosB~\cite{memsic2015telosb} e 
Iris~\cite{memsic2015iris} de propriedade do DCC-UFMG. De todo modo, a prática 
também constará de exibição multimídia dos conceitos aprendidos.} dos conceitos 
discutidos a priore. A parte prática engloba a instalação da pilha de 
protocolos 6LoWPAN/RPL e CoAP em dispositivos (os conhecidos motes). Após o 
entendimento teórico e prático, retomaremos as discussões teóricas sobre as 
implicações e desafios dos momentos apresentados, pontuaremos também as 
padronizações que permitem construir ferramentas de gerenciamento (os 
\textit{middlewares}) dos dispositivos e análise dados, além de destacar 
questões de pesquisa em aberto. Para finalizar, os participantes serão 
convidados a opinar e refletir sobre as tendências futuras da IoT.

\subsection{Perfil do público-alvo}

Este minicurso se destina a profissionais e estudantes de graduação e 
pós-graduação das áreas de Ciência da Computação, Sistemas de Informação, 
Tecnologia da Informação, Engenharias (Elétrica, Automação, Computação) e outras 
correlatas. Alunos ou profissionais com conhecimento em sistemas em rede e/ou 
comunicações digitais podem expandir suas competências com o conteúdo teórico e 
prático discutidos e apresentados ao longo do minicurso. Espera-se que os alunos 
tenham conhecimentos básicos de redes de computadores somente, pois o minicurso 
visa explorar desde os blocos básicos da implementação da IoT até questões mais 
avançadas do tema. Conhecimentos básicos de programação, redes sem fio, sistemas 
distribuídos e análises estatísticas são bem vindos, mas não são obrigatórios 
para o entendimento do curso.


\section{Estrutura prevista do curso}

O conteúdo do minicurso está distribuído nas seguintes seções, que são 
detalhadas a seguir:

\begin{enumerate}
  \item Introdução
    \begin{itemize}
      \item Exemplos de aplicações como um gacho para a motivação da IoT
      \item Enfoque na motivação de conectar dispositivos inteligentes à 
Internet
      \item Sistemas embarcados
      \item Redes de sensores
      \item Computação ubíqua e pervasiva
      \item Definir o que é IoT através da evolução dos sistemas embarcados, 
RSSF e computação ubíqua/pervasiva.
      \item Deixar um gacho com os principais desafios existentes ao nível de 
hardware e ao nível de software
    \end{itemize}

  \item Dispositivos para IoT
    \begin{itemize}
      \item Quais são as melhores referências de dispositivos?
      \item Elencar quais são as principais tecnologias para conectar 
dispositivos
      \item Destacar as limitações dos dispositivos e quais são as principais 
propostas de solução existentes (Ex: energia -> Energy Hasversting)
      \item Custo para manutenção dos dispositivos
      \item Elencar as características dos dispositivos IoT
      \item Detalhar de modo textual e tabular uma lista não exaustiva de 
exemplos de dispositivos.
      \item Construir 'rank' baseado nas características
    \end{itemize}

  \item Software para IoT
    \begin{itemize}
      \item Quais são os sistemas operacionais?
      \item Porque IP para os dispositivos?
	\begin{itemize}
	  \item Quais as adaptações necessárias 6LOWPAN, $\mu$IP...
	\end{itemize}
      \item Paradigmas de comunicação e seus protocolos (facilita o Plug $\& 
$ Play?)
      \begin{itemize}
       \item Muitos-para-um Ex: CTP, MultHopLQI...
       \item Um-para-muitos: Direct Difusion, Deluxe, DIP/DRIP, CodeDRIP....
       \item Qualquer-para-Qualquer: RPL, XCTP, Matrix
      \end{itemize}
      
      \item Modelos de conectividade (Internet of Things X Autornomous Smart 
Objects networks).
      \begin{itemize}
	\item Autornomous Smart Objects networks - objetos que não requerem 
nenhuma conexão com a Intetnet (Ex: smart grids, )
	\item Internet of Things - onde objetos inteligentes realmente estão 
conectados à Internet publica e podem ser acessados diretamente ou através de 
middlewares.
      \end{itemize}

      \item Roteamento
	\begin{itemize}
	 \item XCTP/Matrix (Estudo de caso)
	\end{itemize}

      \item Desenvolvimento (simularoes)
	\item Tabular ou listar os simuladores e as principais características
	\item Exemplificar na prática com o Contiki-RPL 6LowPAN direto do 
simulador para uma rede externa.

      \item Questões de pesquisa
	\begin{itemize}
	  \item Pilha de protocolos X Consumo de recursos
	  \item Segurança*
	\end{itemize}

    \end{itemize}


  \item IoT na prática
  \begin{itemize}
    \item Definir quais serão os experimentos.
    \begin{itemize}
      \item A proposta inicial seria:
      \begin{itemize}
	\item Instalar os códigos de um ou dos SOs TinyOS e Contiki. Porém 
existe o problema de como explicar tal instalação no texto. Além dessa 
demonstração, seria executada consultas aos nós sensores através de requisições 
CoAP.
	\item Realizar uma consulta em uma plataforma middleware tal como o 
João utiliza a Xyvely.
	\item OBS (IMPORTANTE): vale notar que as demonstrações serão 
realizadas ao longo da apresentação e não em um momento específico.
	\item OBS (IMPORTANTE): podemos criar um vídeo para exemplificar de 
forma mais rápida e confiável, mas ainda assim seria interessante levar os 
motes reais para validar o conteúdo do vídeo.
      \end{itemize}
    \end{itemize}

  \end{itemize}

  \item Gerenciamento e análise de dados
    \begin{itemize}
      \item Técnicas para abstrair a heterogeneidade dos dispositivos
	\begin{itemize}
	 \item CoAP, MQTT...
	 \item Ferramentas existentes (Plataformas de middleware)
	\end{itemize}

      \item Trabalhar com dados oriundos da IoT
	\begin{itemize}
	  \item Formato dos dados (JSON, XML ...)
	  \item Aspectos dos dados
	  \begin{itemize}
	    \item Espaços, Correlatos, Diferentes fontes, Imprecisos....
	  \end{itemize}
	  \item Qualidade dos dados (Estudo de caso)
	  \item Fusão de dados (uma questão e 2 níveis de problemas)
	  \begin{itemize}
	    \item com o artigo que o professor passou para Bruno e 
João, ou seja fusão (Estudo de caso).
	    \item in-networks
	  \end{itemize}
	\end{itemize}

      \item Questões de pesquisa
    \end{itemize}
    
  \item Comentários finais
    \begin{itemize}
      \item Questões não comentadas no capítulo (essas questões devem entrar em 
alguma seção?)
      \begin{itemize}
	\item Como tornar os dispositivos Plug $\&$ Play
	\item Localização (quais os problemas existentes? como fazer?)
	\item Segurança
	\item Descoberta de Serviços
	\item Desempenho X quantidades de acessos
	\item IP X Non-IP
	\item Exemplos de Aplicações (Automação residencial, Smart Cities, 
Urban Networks, Monitoramento de Saúde)
      \end{itemize}
      \item Revisão do texto e de forma tabular listar os problemas de pesquisa 
de cada seção.
      \item Agradecimentos
    \end{itemize}


  \item Referências Bibliográficas
\end{enumerate}


\section{Desenvolvimento dos capítulos}
%Para cada tópico/capítulo previsto no item anterior, apresentar um resumo do 
% conteúdo a ser abordado e previsão do número de páginas.

Na \textbf{Introdução (5 páginas)}, é apresentado o que 
são os objetos inteligentes, de onde surgiram e motivações do estudo da IoT. 
Como principal objetivo desta seção tem-se a perspectiva da evolução dos 
sistemas de hardware e software que suportam a IoT dos dias atuais. Os 
principais pontos históricos são levantados, destacando a interseção entre da 
IoT com as Redes de Sensores Sem Fio (RSSF), Redes de Computadores, telefonia 
móvel, computação móvel e ubíqua e, sistemas embarcados. Uma atenção especial é 
dada aos dispositivos IoT (ex: motes de baixo custo, nó sensores comerciais, 
\textit{Arduino} e \textit{Raspberry Pi}) e  as tecnologias mais utilizadas para 
a conexão entre os dispositivos (ex: os padrões IEEE~$802.15.4$, IEEE~$802.11$ e 
\textit{Bluetooth Low Energy (BLE)}). Para concluir a introdução, é feita uma 
visão geral da IoT, aliada a ênfases nos principais desafios do ponto 
de vista de hardware e software que lançam as primeiras questões e direções de 
pesquisa na área. Isto, funcionará como âncora para as próximas seções do 
minicurso.


Na seção \textbf{Dispositivos para IoT (5 páginas)}, são 
discutidos os mais variados dispositivos utilizados na IoT. Classificamos os 
dispositivos em três grandes grupos: i) os \textbf{dispositivos para pesquisa e 
desenvolvimento} são os famosos motes (Mica, Micaz, Telos e outros); ii) 
\textbf{dispositivos de uso doméstico} são aqueles que podem ser usados para 
pesquisa, passatempo (\textit{hobby}) por usuários comuns ou para implementação 
final no caso de empresas, exemplos desses dispositivos são Arduino e 
Raspberry Pi; iii) os \textbf{dispositivos comerciais} são aqueles já em 
produção e venda através de empresas, exemplo desses dispositivos são os Galileo 
e Edison da Intel e Samsung ARTIK.  Em particular, uma ênfase maior é 
realizada ao mote TelosB, o qual é empregado nos exercícios práticos do 
minicurso. Em seguida, são postos em pauta os desafios de pesquisa dos 
dispositivos IoT. Para finalizar a seção, uma conexão com os requisitos de 
software para esses dispositivos é realizada, introduzindo e justificando a 
próxima seção.

A seção \textbf{Software para IoT (15 páginas)} tem por 
objetivo expor a perspectiva de softwares para IoT, tanto para operação quanto 
para desenvolvimento. Nesta seção, a atenção se volta para os sistemas 
operacionais (SO) para IoT e na pilha de protocolos de rede. Inicialmente são 
abordados os SOs TinyOS e Contiki OS, os quais são amplamente empregados em 
dispositivos IoT tanto na academia quanto para produção final de empresas. Em 
seguida, apresentamos os principais simuladores que apoiam o desenvolvimento de 
software para IoT, tais como: NS2/NS3, Cooja, Tossim, OMNet++/Castalia e 
Sinalgo. 

Ainda na seção \textbf{Software para IoT}, damos destaque a variados protocolos 
que podem compor a pilha de protocolos de rede para IoT, em particular, é posto 
em evidência IPv6/6LoWPAN, RPL, CoAP, Zeroconf, MQTT, e as pilhas lwIP e 
$\mu$IP. Além disso, pontuamos o estado da arte em protocolos para IoT que 
exploram questões de endereçamento e paradigmas de comunicação para otimizar a 
rede de objetos IoT, uma amostra desses protocolos são: RPL, CTP, XCTP, MHCL, 
Hydro, OLSR, AODV, DSR e outros que exploram os paradigmas de comunicação para 
otimizar a rede IoT. Como nas seções anteriores, também levantamos questões de 
pesquisa e, principalmente, alertamos para as padronizações que são fundamentais 
para o funcionamento de uma IoT composta por dispositivos heterogêneos. Com base 
nessas considerações e nos conhecimentos discutidos nesta e nas seções 
anteriores, os participantes estão aptos a entender e realizar o exercício 
prático, que é o propósito da próxima seção.

Em \textbf{IoT na prática (8 páginas)}, demonstra-se com
exercícios práticos os conceitos vistos na seções \textit{Dispositivos para IoT 
e Software para IoT}. Esta seção é divida em duas partes. Na primeira, é 
demostrado minunciosamente como instalar a pilha de protocolos com 6LoWPAN/RPL 
em motes TelosB, para tanto, são utilizados os SOs TinyOS e Contiki OS. As 
principais diferenças e interseções das duas diferentes implementações também 
são analisadas. Com isto é possível validar e diagnosticar a operacionalidade 
dos dispositivos através do protocolo ICMPv6. Na segunda parte desta seção, a 
qual tem por objetivo contextualizar e exibir o potencial das padronizações 
sobre o modo de acesso aos recursos da IoT. Para tanto, apresentamos como 
instalar e manipular o protocolo CoAP nos objetos IoT, além de mostrar como 
realizar consultas (inserção, atualização e remoção de estados) nos dispositivos 
da rede. Finalmente chamamos a atenção dos participantes para as diferentes 
aplicações possíveis sobre uma IoT funcional e padronizada, este elo é 
importante para o entendimento da próxima seção que abordará as diferentes 
aplicações e desafios após uma rede de dispositivos IoT funcional.

A seção \textbf{Gerenciamento e análise de dados (15 páginas)} tem por objetivo 
descrever quais os principais desafios ao se implementar aplicações sobre IoT. 
Esta seção é dividida em dois momentos. No primeiro momento, explicamos quais 
são as principais técnicas existentes para abstrair a heterogeneidade dos 
dispositivos IoT. Essas abstrações viabilizam a construção de aplicações e 
serviços sobre uma rede de dispositivos IoT, bem como abstraem as complexidades 
do sistema para usuários finais. Já no segundo momento, é discutido o impacto 
que os dados oriundos da IoT podem causar sobre as aplicações. Neste quesito, 
diversas questões e desafios de pesquisa estão em abertos como, por exemplo, 
tratar imperfeições (imprecisos, vagos, ambíguos etc.) nos dados, como lidar com 
a sua granularidade e inconsistência (ex: dados espaçados no tempo, fora de 
ordem, \textit{outliers}) e, em especial, como trabalhar com dados de fontes 
heterogêneas.

Destacamos quais são os modelos de referências existentes que indicam os 
blocos básicos de construção dessas abstrações, também pontuamos algumas 
plataformas de gerenciamento (\textit{middleware}) e análise de dados 
existentes. Em seguida, retomamos a discussão sobre padronizações e 
desenvolvimento de protocolos otimizados para IoT, dando destaque aos mais 
atuais protocolos MQTT e CoAP. Em seguida, apresentamos ferramentas de gerência 
e análise de dados (Ex: Xively e OpenIoT) e, para os ouvintes ocorrerá uma 
demonstração prática das ferramentas, demonstrando como manipular dados da IoT. 
Para finalizar a seção, levantaremos as principais questões de pesquisa e 
oportunidades relacionadas ao conteúdo apresentado.

Finalmente, na seção \textbf{Conclusões e desafios (2 páginas)}, 
são apresentadas as considerações finais, as conquistas e limitações da IoT nos 
dias atuais, bem como apontamos uma visão do que será IoT no futuro. Chamaremos 
atenção, novamente, dos leitores e participantes do minicurso as questões de 
pesquisa e oportunidades da área.


% \textcolor{red}{
% Divisão do trabalho.
% \begin{itemize}
% \item Introdução (Bruno, Bruna e Professores)
% \item Dispositivos para IoT (Lucas e Bruno)
% \item Software para IoT (Bruna, Bruno e Clayson)
% \item IoT na prática (Bruno, Bruna, Marcos e Luiz)
% \item Gerenciamento e análise de dados (João, Clayson, Bruno, Antonio)
% \item Conclusão (Bruno, Bruna, Lucas e Professores)
% \end{itemize}
% }

% \section{Bibliografia utilizada na preparação do curso}
% Para a confecção do curso são utilizadas como principais fontes as 
% referências:
% \linebreak \cite{BorgesNeto2015}, \cite{shelby2014constrained}, 
% \cite{winter2012rpl}, \cite{pires2015plataformas}, \linebreak 
% \cite{chaouchi2013internet}, \cite{vasseur2010interconnecting}, 
% \cite{Khaleghi2013Multi}, \cite{ludovici2013tinycoap}, \linebreak 
% \cite{ko2011connecting}, \cite{vieira2010redes}, \cite{Gubbi2013}.


\nocite{*}
\bibliographystyle{sbc}
\bibliography{refs}


\section{Currículo resumido dos autores} 

\begin{itemize}

\item \textbf{Bruno P. Santos}: possui graduação em Bacharelado em 
Ciência da Computação Universidade Estadual de Santa Cruz - UESC (2012), mestre 
em Ciência da Computação pela Universidade Federal de Minas Gerais - UFMG (2015) 
e, atualmente, é aluno de doutorado pela Universidade Federal de Minas Gerais - 
UFMG (2015 - 2019). As principais áreas de interesse são redes de computadores, 
gerenciamento de redes, redes de sensores sem fio, computação móvel, computação 
ubíqua, redes inteligentes, fusão/análise de dados, Modelagem Computacional para 
CAD (Computação de Alto Desempenho), processamento paralelo (GPU, MPI, OpenMP). 
Atuando ativamente no desenvolvimento de técnica e tecnologias nas áreas de 
interesse. 

\item \textbf{Lucas A. M. Silva}: possui graduação em Ciência da Computação 
pela Universidade Federal de Minas Gerais - UFMG (2014), atualmente, é aluno de 
mestrado pela Universidade Federal Minas Gerais - UFMG (2014 - 2016). As 
principais áreas de pesquisa são redes de computadores, Redes Definidas por 
Software e Internet das Coisas, participando ativamente em projetos nas áreas 
citadas.

\item \textbf{Bruna S. Peres}: possui graduação em Ciência da Computação pela 
Pontifícia Universidade Católica de Minas Gerais - PUC Minas (2012), mestre em 
Ciência da Computação pela Universidade Federal de Minas Gerais - UFMG (2015) e, 
atualmente, é aluna de doutorado pela Universidade Federal de Minas Gerais - 
UFMG (2015 - 2019). As principais áreas de interesse são redes de computadores, 
redes de sensores sem fio, computação móvel, computação ubíqua, modelos de 
conectividade e redes inteligentes. Atuando ativamente em projetos nas áreas de 
interesse. 

\item \textbf{Clayson S. F. S. Celes}: Estudante de doutorado em ciência da 
computação na Universidade Federal de Minas Gerais (UFMG). Possui mestrado em 
Ciência da Computação pela UFMG. Graduação em Ciência da Computação pela 
Universidade Estadual do Ceará. Graduação em Teleinformática pelo Instituto 
Federal de Educação, Ciência e Tecnologia do Ceará. Tem experiência na área de 
Ciência da Computação, com ênfase em sistemas distribuídos, atuando 
principalmente nos seguintes temas: computação móvel e ubíqua, algoritmos 
distribuídos, comunicação sem fio e redes de computadores.

\item \textbf{João B. B. Neto}: Possui mestrado em Engenharia de Teleinformática 
pela Universidade Federal do Ceará (2009) e graduação em Ciência da Computação 
pela Universidade do Estado do Rio Grande do Norte (2006). Atualmente é e 
doutorando em Ciência da Computação pela Universidade Federal de Minas Gerais e 
professor assistente na Universidade Federal do Rio Grande do Norte, onde atua 
em projetos de desenvolvimento, pesquisa e extensão. Tem experiência na área de 
Ciência da Computação, com ênfase em Redes de Computadores, atuando 
principalmente nos seguintes temas: Redes de Sensores Sem Fio, Computação Móvel 
e Computação Ubíqua. Também atua nos seguintes temas: Sistemas Operacionais, 
Sistemas Distribuídos, Aplicações Web e Software Livre.

\item \textbf{Olga N. Goussevskaia}: Possui doutorado em ciência da 
computação pelo Instituto Federal de Tecnologia Suíço (ETH Zurich, 2009) e 
mestrado em ciência da computação pela Universidade Federal de Minas Gerais 
(UFMG, 2005). Atualmente é professora adjunto na Universidade Federal de Minas 
Gerais. Tem experiência na área de ciência da computação, com ênfase em 
algoritmos para redes de comunicação sem fio, atuando principalmente nos 
seguintes temas: complexidade computacional e algoritmos de escalonamento de 
requisições, algoritmos distribuídos, modelagem de interferência, assim como em 
temas como análise de redes complexas e aplicativos baseados em dados coletados 
na web.

\item \textbf{Marcos A. M. Vieira}: possui graduação em Bacharelado em Ciência 
da Computação pela Universidade Federal de Minas Gerais(2002), mestrado em 
Ciências da Computação pela Universidade Federal de Minas Gerais(2004), mestrado 
em Computer Science pela University of Southern California(2007), doutorado em 
Ciência da Computação pela University of Southern California(2010) e 
pós-doutorado pela Universidade Federal de Minas Gerais(2011). Atualmente é 
Professor Adjunto da Universidade Federal de Minas Gerais e bolsista de 
produtividade nível 2 no CNPq. Tem experiência na área de Ciência da Computação, 
com ênfase em Sistemas de Computação, atuando principalmente nos temas Rede de 
Sensores Sem Fio e Robótica.

\item \textbf{Luiz F. M. Vieira}: possui graduação em Ciência da Computação 
pela Universidade Federal de Minas Gerais(2002), mestrado em Ciências da 
Computação pela Universidade Federal de Minas Gerais(2004), mestrado em Computer 
Science pela Computer Science Department - UCLA(2007), doutorado em Computer 
Science pela Computer Science Department - UCLA(2009) e pós-doutorado pela 
Universidade Federal de Minas Gerais(2010). Atualmente é Professor Adjunto da 
Universidade Federal de Minas Gerais, bolsista de produtividade nível 2 no CNPq, 
Revisor de periódico da IEEE Transactions on Wireless Communications, Revisor de 
periódico da IEEE Transactions on Mobile Computing, Revisor de periódico da 
IEEE/ACM Transactions on Networking (Print), Revisor de periódico da Ad Hoc 
Networks e Revisor de periódico da IEEE Communications Letters (Print).

\item \textbf{Antonio A. F. Loureiro}: possui graduação em Ciência da Computação 
pela Universidade Federal de Minas Gerais (1983), mestrado em Ciência da 
Computação pela Universidade Federal de Minas Gerais (1987) e doutorado em 
Ciência da Computação pela University of British Columbia, Canadá (1995). 
Atualmente é Professor Titular do Departamento de Ciência da Computação da 
Universidade Federal de Minas Gerais. Tem experiência na área de Ciência da 
Computação, com ênfase em sistemas distribuídos, atuando principalmente nos 
seguintes temas: algoritmos distribuídos, computação móvel/ubíqua, comunicação 
sem fio, gerenciamento de redes, redes de computadores, redes de sensores sem 
fio.

\end{itemize}


\end{document}

