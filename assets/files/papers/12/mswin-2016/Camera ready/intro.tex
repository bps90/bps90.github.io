\section{Introduction}

IPv6 over Low-power Wireless Personal Area Networks
(6LoWPAN) is a working group
inspired by the idea that even the smallest low-power devices should
be able to run the Internet Protocol to become part of the Internet
of Things. Standard routing protocols for 6LowPAN, such as CTP
(Collection Tree Protocol~\cite{Fonseca:2009}) and RPL (IPv6 Routing
Protocol for Low-Power and Lossy Networks~\cite{rfc6550}), have two
distinctive characteristics: communication devices use unstructured
IPv6 addresses that do not reflect the topology of the network
(typically derived from their MAC addresses), and routing lacks
support for any-to-any communication since it is based on
distributed collection tree structures focused on bottom-up data
flows (from the leaves to the root).
The problem with such one-directional routing is that it makes it
inefficient to build important network functions, such as
configuration routines and reliable mechanisms to ensure the
delivery of end-to-end data.

Even though CTP does not support any-to-any traffic, the specification of
RPL defines two modes of operation for top-down data flows: the non-storing
mode, which uses source routing, and the storing mode, in which each node
maintains a routing table for all possible destinations. This requires $O(n)$
space (where $n$ is the total number of nodes), which is unfeasible for
memory-constrained devices.

Some works have addressed this problem from different perspectives.
In \cite{2016techreport}, the authors proposed a hierarchical IPv6 address
allocation scheme, referred to as MHCL, to enable any-to-any routing
by incorporating network topology information into the IPv6 address,
assigned to each node in a multihop fashion. MHCL was implemented as
a subroutine of RPL and showed to enable any-to-any routing using
compact routing tables. MHCL stores the IPv6 address range of the
entire subtree rooted at a child node in a single routing table
entry, which results in $O(k)$ memory space, where $k$ is the number
of one-hop descendants of a node in the collection tree. The
downside of MHCL is that it was not designed to deal with network faults
and dynamic network topologies since a message can be dropped
whenever a node or link failure occurs in the address hierarchy.

In this work, we build upon the idea of using hierarchical IPv6
address allocation and propose Matrix, a routing scheme for dynamic
network topologies and fault-tolerant any-to-any data flows in
6LoWPAN. Matrix assumes there is an underlying collection tree
topology (provided by CTP or RPL, for instance), in which nodes have
static locations, i.e., are not mobile, and links are dynamics,
i.e., nodes might choose different parents according to link quality
dynamics. Matrix uses only one-hop information in the routing tables
and implements a local broadcast mechanism to forward messages to
the right subtree when node or link failures occur.

After the network has been initialized and all nodes
have received an IPv6 address range, three simultaneous distributed trees are
maintained by all nodes:
the collection tree (Ctree), the IPv6 address tree (IPtree), and the
reverse collection tree (RCtree).
Initially, any-to-any packet forwarding is performed using Ctree for
bottom-up and IPtree for top-down data flows. Whenever a node or
link fails or Ctree changes, the new link is added in the reverse
direction into RCtree and is maintained as long as this topology
change persists. Top-down data packets are then forwarded from
IPtree to RCtree via a local broadcast. The node that receives a
local-broadcast checks whether it knows the subtree of the
destination IPv6 address: if yes then it forwards the packet to the
right subtree via RCtree and the packet continues its path in the
IPtree until the final destination.


Why is this approach robust to network dynamics? Routing is
performed using the address hierarchy represented by the IPtree, so
whenever a link or node fails, messages addressed to destinations in
the corresponding subtree may be lost. Matrix uses the (dynamic)
reverse collection tree and the local broadcast mechanism to forward
messages to the right subtree, as long as an alternative route
exists. Note that this local rerouting mechanism does not
guarantee that all messages will be delivered. We argue that the probability that the message
will be forwarded to the appropriate subtree is
high, as long as there is a valid path, due to the geometric properties of wireless
networks. Our simulations showed that this intuition is, in fact, correct.


Why does this approach scale? Each node stores only one-hop
neighborhood information, namely: the id of its parent in Ctree, the
IPv6 address ranges of its children in the IPtree, and the IPv6
address ranges of its (temporary) children in the RCtree. Therefore,
the memory footprint at each node is $O(k)$, where $k$ is the number
of children at any given moment in time. The impact of such low
memory footprint on the end-to-end routing success is impressive:
whereas RPL delivers less than 20\% of packets in some
scenarios, Matrix delivers 99\% of packets
successfully, without end-to-end mechanisms.


We evaluated the
proposed protocol both analytically and by simulation. Even though Matrix is
platform-independent, we implemented it as a subroutine of CTP on TinyOS and conducted
simulations on TOSSIM.
The results showed that, when it comes to any-to-any communication, Matrix
presents significant gains in terms of reliability (higher any-to-any message
delivery) and scalability (presenting a
constant, as opposed to linear, memory complexity at each node) at a moderate
cost of additional control messages, when compared to other state-of-the-art
protocols, such as RPL. 


To sum up, Matrix achieves the following essential goals that motivated our work:
\begin{itemize}
  \item \textbf{Any-to-any routing}: Matrix enables end-to-end connectivity
  between hosts located within or outside the 6LoWPAN.
  \item \textbf{Memory efficiency}: Matrix uses compact routing tables and,
  therefore, is scalable to very large networks.
  \item \textbf{Reliability}: Matrix achieves $99\%$ delivery
  without end-to-end mechanisms, and delivers $\geq 95\%$ of
  end-to-end packets when a route exists under challenging network conditions.
  \item \textbf{Communication efficiency}: Matrix uses adaptive beaconing based
  on Trickle algorithm \cite{Levis:2004} to minimize the number of control
  messages in dynamic network topologies (except with node mobility).
  \item \textbf{Hardware independence}: Matrix does not rely on specific radio
  chip features, and only assumes an underlying collection tree structure.
  \item \textbf{IoT integration}: Matrix allocates global (and structured) IPv6
addresses to all nodes, which allow nodes to act as destinations integrated into
the Internet, contributing to the realization of the Internet of Things.
\end{itemize}


The rest of this paper is organized as follows. In
Section~\ref{sec:matrix} we describe the Matrix protocol design. In
Section~\ref{sec:analysis}, we analyze the message complexity of the protocol.
In Section~\ref{sec:results} we present our analytical and simulation results.
In Section~\ref{sec:related} we discuss some related work. Finally, in Section~\ref{sec:conclusion} we present the concluding remarks.
