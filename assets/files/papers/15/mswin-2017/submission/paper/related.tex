\section{Related work}
\label{sec:related-work}

In the world of tiny (IoT) several mobility-enabling routing protocols have been proposed. Firstly we highlight $\mu$Matrix's features against its original static version \cite{Peres:2016}. Then, we survey recent protocols in the context of 6LoWPAN and put them in perspective with $\mu$Matrix. 
  
Matrix was originally proposed without support for mobility\cite{Peres:2016}. If a node moved from its home location, the hierarchical IPv6 address allocation would become invalid and compromise downward routing. Although RPL~\cite{rfc6550} is the standard protocol for 6LoWPANs, it presents limitations, for example, in mobility scenarios, scalability issues, reliability and robustness for point-to-multipoint traffic~\cite{iova2016rpl, Peres:2016}. Most recent mobile-enabled routing protocols are RPL extensions. They deal with mobile issues, but they do not handle RPL drawbacks. Co-RPL~\cite{gaddour2014co} provides mobility support to RPL but without Trickle. This turns Co-RPL more responsive but has higher overhead. MMRPL~\cite{cobarzan2014analysis} modifies the RPL beacon periodicity by replacing the Trickle mechanism with a Reverse Trickle-Like. Their Reverse Trickle decays exponentially, while our approach quickly goes to the minimum after an unacknowledged beacon. MMRPL also needs some static nodes. In ME-RPL~\cite{el2012mobility}, static nodes have higher priority than mobile ones. ME-RPL requires some fixed nodes. The memory requirement to downward routes is still prohibitive. mRPL~\cite{fotouhi2015mrpl} proposes a hand-off mechanism for mobile nodes in RPL by separating nodes into mobile (MN) or serving access point (AP). They use smart-HOP algorithm on MN nodes to perform hand-off between AP. 

XCTP~\cite{santos2015extend} extends CTP to support bidirectional traffic. XCTP does not support IPv6 addressing and any-to-any traffic. Hydro~\cite{dawson2010hydro} fills the gap of any-to-any traffic, but it requires static nodes with a large memory to perform the routing and support mobility nodes.

Mobile IP~\cite{perkins2011mobility} and Hierarchical Mobile IPv6 (HMIPv6) Mobility Management~\cite{rfc5380} are standards for IPv6 networks for handling local mobility. However, they are not designed for 6LoWPANs, they do not present a mobility detection or adjustable timers. LOAD~\cite{kim20076lowpan} and DYMO-Low~\cite{kim2007dynamic} are 6LoWPAN protocols inspired in AODV and DYMO, but they are not suitable for mobile networks.

Table~\ref{tab:rel-works} summarizes properties of the related protocols.

\begin{table}[!t]
\centering

\caption{Routing protocol properties}
\label{tab:rel-works}
\resizebox{\columnwidth}{!}{%
\begin{tabular}{@{}lccccccccc@{}}
\toprule
\textbf{Feature}    & \textbf{$\mu$Matrix}                                                              & \textbf{RPL}   & \textbf{Co-RPL}                                          & \textbf{MMRPL}                                                  & \textbf{ME-RPL}                                             & \textbf{mRPL}                                               & \textbf{DMR}                                                & \textbf{Hydro}                                              & \textbf{XCTP}  \\ \midrule
Bottom-p            & \CheckmarkBold                                                                    & \CheckmarkBold & \CheckmarkBold                                           & \CheckmarkBold                                                  & \CheckmarkBold                                              & \CheckmarkBold                                              & \CheckmarkBold                                              & \CheckmarkBold                                              & \CheckmarkBold \\
Top-down            & \CheckmarkBold                                                                    & \CheckmarkBold & \CheckmarkBold                                           & \CheckmarkBold                                                  & \CheckmarkBold                                              & \CheckmarkBold                                              &                                                             & \CheckmarkBold                                              & \CheckmarkBold \\
Any-to-any          & \CheckmarkBold                                                                    & \CheckmarkBold & \CheckmarkBold                                           & \CheckmarkBold                                                  & \CheckmarkBold                                              & \CheckmarkBold                                              &                                                             & \CheckmarkBold                                              &                \\
Address Allocation  & \CheckmarkBold                                                                    &                &                                                          &                                                                 &                                                             &                                                             &                                                             &                                                             &                \\
IPv6 support        & \CheckmarkBold                                                                    & \CheckmarkBold & \CheckmarkBold                                           & \CheckmarkBold                                                  & \CheckmarkBold                                              & \CheckmarkBold                                              & \CheckmarkBold                                              &                                                             &                \\
Memory efficiency   & \CheckmarkBold                                                                    &                &                                                          &                                                                 &                                                             &                                                             &                                                             &                                                             &                \\
Fault Tollerance    & \CheckmarkBold                                                                    &                &                                                          &                                                                 &                                                             &                                                             &                                                             &                                                             & \CheckmarkBold \\
Local Repair        & \CheckmarkBold                                                                    &                &                                                          &                                                                 &                                                             &                                                             &                                                             &                                                             &                \\
Topological changes & \begin{tabular}[c]{@{}c@{}}Reverse\\ Trickle\end{tabular}                         & Trickle        & \begin{tabular}[c]{@{}c@{}}Periodic\\ fixed\end{tabular} & \begin{tabular}[c]{@{}c@{}}Reverse \\ Trickle-like\end{tabular} & Trickle                                                     & Trickle                                                     & Trickle                                                     & Periodic fixed                                              & Trickle        \\
Constraints         & \begin{tabular}[c]{@{}c@{}}Nodes should\\ return to \\ home location\end{tabular} &                &                                                          & \begin{tabular}[c]{@{}c@{}}Need static\\ nodes\end{tabular}     & \begin{tabular}[c]{@{}c@{}}Need static\\ nodes\end{tabular} & \begin{tabular}[c]{@{}c@{}}Need static\\ nodes\end{tabular} & \begin{tabular}[c]{@{}c@{}}Need static\\ nodes\end{tabular} & \begin{tabular}[c]{@{}c@{}}Need static\\ nodes\end{tabular} &                \\ \bottomrule
\end{tabular}%
}
\end{table}